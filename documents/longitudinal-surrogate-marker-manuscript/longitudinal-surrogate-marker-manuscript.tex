\documentclass[useAMS,usenatbib,referee]{biom}
%\documentclass[useAMS,usenatbib,referee]{biom}
%
%
%  Papers submitted to Biometrics should ALWAYS be prepared
%  using the referee option!!!!
%
%
% If your system does not have the AMS fonts version 2.0 installed, then
% remove the useAMS option.
%
% useAMS allows you to obtain upright Greek characters.
% e.g. \umu, \upi etc.  See the section on "Upright Greek characters" in
% this guide for further information.
%
% If you are using AMS 2.0 fonts, bold math letters/symbols are available
% at a larger range of sizes for NFSS release 1 and 2 (using \boldmath or
% preferably \bmath).
%
% The usenatbib command allows the use of Patrick Daly's natbib package for
% cross-referencing.
%
% If you wish to typeset the paper in Times font (if you do not have the
% PostScript Type 1 Computer Modern fonts you will need to do this to get
% smoother fonts in a PDF file) then uncomment the next line
% \usepackage{Times}
%%%%% AUTHORS - PLACE YOUR OWN MACROS HERE %%%%%

\usepackage[figuresright]{rotating}
\usepackage{tikz}
\usepackage{amsmath}
\usepackage[hyphens]{url} % not crucial - just used below for the URL
\usepackage{hyperref}
\usepackage[utf8]{inputenc}
\usepackage{graphicx}
\usepackage{longtable}
\usepackage{booktabs}
%% \raggedbottom % To avoid glue in typesetteing, sbs>>

\usepackage{color}
\usepackage{fancyvrb}
\newcommand{\VerbBar}{|}
\newcommand{\VERB}{\Verb[commandchars=\\\{\}]}
\DefineVerbatimEnvironment{Highlighting}{Verbatim}{commandchars=\\\{\}}
% Add ',fontsize=\small' for more characters per line
\usepackage{framed}
\definecolor{shadecolor}{RGB}{248,248,248}
\newenvironment{Shaded}{\begin{snugshade}}{\end{snugshade}}
\newcommand{\KeywordTok}[1]{\textcolor[rgb]{0.13,0.29,0.53}{\textbf{#1}}}
\newcommand{\DataTypeTok}[1]{\textcolor[rgb]{0.13,0.29,0.53}{#1}}
\newcommand{\DecValTok}[1]{\textcolor[rgb]{0.00,0.00,0.81}{#1}}
\newcommand{\BaseNTok}[1]{\textcolor[rgb]{0.00,0.00,0.81}{#1}}
\newcommand{\FloatTok}[1]{\textcolor[rgb]{0.00,0.00,0.81}{#1}}
\newcommand{\ConstantTok}[1]{\textcolor[rgb]{0.00,0.00,0.00}{#1}}
\newcommand{\CharTok}[1]{\textcolor[rgb]{0.31,0.60,0.02}{#1}}
\newcommand{\SpecialCharTok}[1]{\textcolor[rgb]{0.00,0.00,0.00}{#1}}
\newcommand{\StringTok}[1]{\textcolor[rgb]{0.31,0.60,0.02}{#1}}
\newcommand{\VerbatimStringTok}[1]{\textcolor[rgb]{0.31,0.60,0.02}{#1}}
\newcommand{\SpecialStringTok}[1]{\textcolor[rgb]{0.31,0.60,0.02}{#1}}
\newcommand{\ImportTok}[1]{#1}
\newcommand{\CommentTok}[1]{\textcolor[rgb]{0.56,0.35,0.01}{\textit{#1}}}
\newcommand{\DocumentationTok}[1]{\textcolor[rgb]{0.56,0.35,0.01}{\textbf{\textit{#1}}}}
\newcommand{\AnnotationTok}[1]{\textcolor[rgb]{0.56,0.35,0.01}{\textbf{\textit{#1}}}}
\newcommand{\CommentVarTok}[1]{\textcolor[rgb]{0.56,0.35,0.01}{\textbf{\textit{#1}}}}
\newcommand{\OtherTok}[1]{\textcolor[rgb]{0.56,0.35,0.01}{#1}}
\newcommand{\FunctionTok}[1]{\textcolor[rgb]{0.00,0.00,0.00}{#1}}
\newcommand{\VariableTok}[1]{\textcolor[rgb]{0.00,0.00,0.00}{#1}}
\newcommand{\ControlFlowTok}[1]{\textcolor[rgb]{0.13,0.29,0.53}{\textbf{#1}}}
\newcommand{\OperatorTok}[1]{\textcolor[rgb]{0.81,0.36,0.00}{\textbf{#1}}}
\newcommand{\BuiltInTok}[1]{#1}
\newcommand{\ExtensionTok}[1]{#1}
\newcommand{\PreprocessorTok}[1]{\textcolor[rgb]{0.56,0.35,0.01}{\textit{#1}}}
\newcommand{\AttributeTok}[1]{\textcolor[rgb]{0.77,0.63,0.00}{#1}}
\newcommand{\RegionMarkerTok}[1]{#1}
\newcommand{\InformationTok}[1]{\textcolor[rgb]{0.56,0.35,0.01}{\textbf{\textit{#1}}}}
\newcommand{\WarningTok}[1]{\textcolor[rgb]{0.56,0.35,0.01}{\textbf{\textit{#1}}}}
\newcommand{\AlertTok}[1]{\textcolor[rgb]{0.94,0.16,0.16}{#1}}
\newcommand{\ErrorTok}[1]{\textcolor[rgb]{0.64,0.00,0.00}{\textbf{#1}}}
\newcommand{\NormalTok}[1]{#1}

% tightlist command for lists without linebreak
\providecommand{\tightlist}{%
  \setlength{\itemsep}{0pt}\setlength{\parskip}{0pt}}
%%%%%%%%%%%%%%%%%%%%%%%%%%%%%%%%%%%%%%%%%%%%%%%%

\setcounter{footnote}{2}

\title[]{Longitudinal surrogate marker analysis}

\author{ Denis Agniel \email{\href{mailto:dagniel@rand.org}{\nolinkurl{dagniel@rand.org}}} \\ RAND Corporation  \and
		 Layla Parast \email{\href{mailto:parast@rand.org}{\nolinkurl{parast@rand.org}}} \\ RAND Corporation 
	   }

\begin{document}

\date{{\it Received Mar} 2019}

\pagerange{\pageref{firstpage}--\pageref{lastpage}} \pubyear{2019}

\volume{0}
\artmonth{January}
\doi{0000-0000-0000}

%  This label and the label ``lastpage'' are used by the \pagerange
%  command above to give the page range for the article

\label{firstpage}

%  pub the summary here

\begin{abstract}
The text of your summary. Should not exceed 225 words.
\end{abstract}

%
%  Please place your key words in alphabetical order, separated
%  by semicolons, with the first letter of the first word capitalized,
%  and a period at the end of the list.
%

\begin{keywords}
longitudinal data; surrogate markers; nonparametric analysis.
\end{keywords}

\maketitle

\input{Macro} \input{GrandMacros} \def\sone{^{(1)}} \def\szero{^{(0)}}

\section{Introduction}\label{intro}

\section{Method}\label{sec:1}

Let the data for analysis consist of \(n\) independent observations of
the form \((Y_i, \bX_i, A_i)_{i=1, ..., n}\), \(A_i\) represents an
indicator for treatment or intervention,
\(\bX_i = (X_{ij})_{j=1, ... n_i}\) is a longitudinally collected
surrogate marker, and \(Y_i\) is a primary outcome of interest, all for
subject \(i\). We assume for simplicity of presentation that patients
are randomly assigned at baseline to treatment or control and that \(Y\)
is fully observed. We further assume that there exists \(X(\cdot)\) an
underlying surrogate marker trajectory, which we only observe \(n_i\)
times, possibly at only a few, irregularly spaced times and with error.

Furthermore, let \(Y_i\sone\) and \(Y_i\szero\) denote the primary
outcome one would observe if, possibly contrary to fact, subject \(i\)
received treatment and control, respectively. We assume the stable unit
treatment value assumption (SUTVA, \citet{rosenbaum1983central}).
Similarly, let \(X_i\sone\) and \(X_i\szero\) denote the summary markers
under treatment and control. We assume that the joint distribution of
\(Y_i\) and \(\bX_i\) is given by \(f_j(y, \bx) = f_j(y|\bx)g_j(\bx)\)
in treatment group \(j\) where \(f_j(y|\bx)\) is the density of \(Y\)
conditional on \(\bX = \bx\) and \(g_j(\bx)\) is the density function
for \(\bX_i\) in group \(D = j\).

We are interested in estimating the proportion of treatment effect on
the primary outcome that is explained by the longitudinal surrogate
marker. We define the overall treatment effect, \(\Delta\), as the
expected difference in \(Y\) under treatment and control,
\[\Delta=E(Y\sone-Y\szero).\] Because of randomization, we can use the
observed data to estimate \(\Delta\)
\[E[Y | A = 1] - E[Y | A = 0] = \int y f_0(y|\bx)g_0(\bx)dyd\bx - \int y f_1(y|\bx)g_1(\bx)dyd\bx.\]
We aim to measure the surrogate value of \(\bX\) comparing \(\Delta\) to
the residual treatment effect that would be observed if the \(\bX\) was
distributed the same in both groups. In the context of a scalar \(X\),
the residual treatment effect can be defined as

\section{Simulation studies}\label{sec:2}

\section{Analysis of longitudinal CD4 count
surrogacy}\label{analysis-of-longitudinal-cd4-count-surrogacy}

\section{Discussion}\label{discussion}

\section{Figures and tables}\label{figures-and-tables}

\subsection{Figures coming from R}\label{figures-coming-from-r}

\paragraph{Normal figure embedded in
text}\label{normal-figure-embedded-in-text}

\begin{verbatim}
## Warning in plot.formula(runif(25) ~ runif(25)): the formula 'runif(25) ~
## runif(25)' is treated as 'runif(25) ~ 1'
\end{verbatim}

\begin{figure}
\centering
\includegraphics{longitudinal-surrogate-marker-manuscript_files/figure-latex/fig2-1.pdf}
\caption{Output from \texttt{pdf()}}
\end{figure}

\clearpage

\subsection{Tables coming from R}\label{tables-coming-from-r}

\begin{Shaded}
\begin{Highlighting}[]
\KeywordTok{print}\NormalTok{(xtable}\OperatorTok{::}\KeywordTok{xtable}\NormalTok{(}\KeywordTok{head}\NormalTok{(mtcars)[,}\DecValTok{1}\OperatorTok{:}\DecValTok{4}\NormalTok{], }
\DataTypeTok{caption =} \StringTok{"Caption centered under table"}\NormalTok{, }\DataTypeTok{label =} \StringTok{"tab1"}\NormalTok{), }
\DataTypeTok{comment =} \OtherTok{FALSE}\NormalTok{, }\DataTypeTok{timestamp =} \OtherTok{FALSE}\NormalTok{, }\DataTypeTok{caption.placement =} \StringTok{"top"}\NormalTok{)}
\end{Highlighting}
\end{Shaded}

\begin{table}[ht]
\centering
\caption{Caption centered under table} 
\label{tab1}
\begin{tabular}{rrrrr}
  \hline
 & mpg & cyl & disp & hp \\ 
  \hline
Mazda RX4 & 21.00 & 6.00 & 160.00 & 110.00 \\ 
  Mazda RX4 Wag & 21.00 & 6.00 & 160.00 & 110.00 \\ 
  Datsun 710 & 22.80 & 4.00 & 108.00 & 93.00 \\ 
  Hornet 4 Drive & 21.40 & 6.00 & 258.00 & 110.00 \\ 
  Hornet Sportabout & 18.70 & 8.00 & 360.00 & 175.00 \\ 
  Valiant & 18.10 & 6.00 & 225.00 & 105.00 \\ 
   \hline
\end{tabular}
\end{table}

Table \ref{tab1} shows these numbers. Some of those numbers are plotted
in Figure \ref{fig:fig1}.

\begin{Shaded}
\begin{Highlighting}[]
\KeywordTok{head}\NormalTok{(mtcars[,}\DecValTok{1}\OperatorTok{:}\DecValTok{4}\NormalTok{])}
\end{Highlighting}
\end{Shaded}

\begin{verbatim}
##                    mpg cyl disp  hp
## Mazda RX4         21.0   6  160 110
## Mazda RX4 Wag     21.0   6  160 110
## Datsun 710        22.8   4  108  93
## Hornet 4 Drive    21.4   6  258 110
## Hornet Sportabout 18.7   8  360 175
## Valiant           18.1   6  225 105
\end{verbatim}


\bibliographystyle{biom}
\bibliography{bibliography.bib}


\label{lastpage}

\end{document}
